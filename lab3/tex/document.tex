\documentclass[a4paper,12pt]{article}

\usepackage[T2A]{fontenc}			
\usepackage[utf8]{inputenc}			
\usepackage[english,russian]{babel}	

\usepackage[
bookmarks=true, colorlinks=true, unicode=true,
urlcolor=black,linkcolor=black, anchorcolor=black,
citecolor=black, menucolor=black, filecolor=black,
]{hyperref}

\usepackage{color}
\usepackage{caption}
\DeclareCaptionFont{white}{\color{black}}
\DeclareCaptionFormat{listing}{\colorbox{white}{\parbox{\textwidth}{#1#2#3}}}
\captionsetup[lstlisting]{format=listing,labelfont=white,textfont=white}

\usepackage{amsmath,amsfonts,amssymb,amsthm,mathtools} 
\usepackage{wasysym}

\usepackage{graphicx}
%\usepackage[cache=false]{minted}
\usepackage{cmap}
\usepackage{indentfirst}

\usepackage{listings} 
\usepackage{fancyvrb}

\usepackage{geometry}
\geometry{left=2cm}
\geometry{right=1.5cm}
\geometry{top=1cm}
\geometry{bottom=2cm}

\setlength{\parindent}{5ex}
\setlength{\parskip}{0.5em}

\usepackage{pgfplots}
\usetikzlibrary{datavisualization}
\usetikzlibrary{datavisualization.formats.functions}

\begin{document}
	\lstset{ %
		language=C,                 % выбор языка для подсветки (здесь это С)
		basicstyle=\small\sffamily, % размер и начертание шрифта для подсветки кода
		numbers=left,               % где поставить нумерацию строк (слева\справа)
		numberstyle=\tiny,           % размер шрифта для номеров строк
		stepnumber=1,                   % размер шага между двумя номерами строк
		numbersep=5pt,                % как далеко отстоят номера строк от подсвечиваемого кода
		backgroundcolor=\color{white}, % цвет фона подсветки - используем \usepackage{color}
		showspaces=false,            % показывать или нет пробелы специальными отступами
		showstringspaces=false,      % показывать или нет пробелы в строках
		showtabs=false,             % показывать или нет табуляцию в строках
		frame=single,              % рисовать рамку вокруг кода
		tabsize=2,                 % размер табуляции по умолчанию равен 2 пробелам
		captionpos=t,              % позиция заголовка вверху [t] или внизу [b] 
		breaklines=true,           % автоматически переносить строки (да\нет)
		breakatwhitespace=false, % переносить строки только если есть пробел
		escapeinside={\%*}{*)}   % если нужно добавить комментарии в коде
	}
	
	% Титульный лист
	\begin{figure}[h!]
		\begin{center}
			{\includegraphics[scale = 0.4]{titul.jpg}}
			\label{titul}
		\end{center}
	\end{figure}
	
	\vspace*{15mm} 
	
	\huge
	\begin{center}
		Дисциплина: <<Функциональное и логическое программирование>>
	\end{center}
	\vspace*{15mm} 	
	
	\begin{center}
		Лабораторная работа №3
	\end{center}
	
	\vspace*{15mm} 	
	
	\large
	\begin{flushright}
		Студент: Левушкин И. К. \\
		Группа: ИУ7-62Б \\
		Преподаватели: Толпинская Н. Б., \\ Строганов Ю. В. \\
	\end{flushright}
	
	\vspace*{30mm}
	\begin{center}
		Москва, 2020 г.  
	\end{center}
	\thispagestyle{empty}
	
	
	\newpage
	
	\section{Составить диаграмму вычисления следующих выражений:}
	\textit{(equal 3 (abs - 3))}
	\vspace*{60mm} 
	
	\textit{(equal (+ 1 2) 3)}
	\vspace*{60mm} 
	
	\textit{(equal (* 4 7) 21)}
	\vspace*{60mm} 
	
	\textit{(equal (* 2 3) (+ 7 2))}
	\vspace*{60mm} 
	
	\textit{(equal (- 7 3) (* 3 2))}
	\vspace*{60mm} 
	
	\textit{(equal (abs (- 2 4)) 3)}
	\vspace*{60mm} 
	
	\newpage
	
	\section{Написать функцию, вычисляющую гипотенузу прямоугольного
треугольника по заданным катетам и составить диаграмму её вычисления.
	}
	\textit{(defun f1 (a b) (sqrt (+ (* a a) (* b b) ) ) )}
	\newpage

	\section{Написать функцию, вычисляющую объем параллелепипеда по 3-м его сторонам, и составить диаграмму ее вычисления.}
	\textit{(defun v (x y z) (* x y z))}
	\newpage
	
	\section{Каковы результаты вычисления следующих выражений?}
	
	\begin{table} [h!]
		\begin{center}
			\begin{tabular}{|l|l|}
				\hline
				{\bf  Выражение} &    {\bf Результат} \\
				\hline
				{(list 'a c)} & error: variable C has no value\\
				\hline
				{(cons 'a (b c))} & error: undefined function B\\
				\hline
				{(cons 'a '(b c))} & (A B C)\\
				\hline
				{(caddr (1 2 3 4 5))} & error: 1 is not a function name\\
				\hline
				{(cons'a'b'c)} & error: too many arguments given to CONS\\
				\hline
				{(list 'a (b c))} & error: undefined function B \\
				\hline
				{(list a '(b c))} & error: variable A has no value\\
				\hline
				{(list (+ 1 '(length '(1 2 3))))} & error: +: (LENGTH '(1 2 3)) is not a number \\
				\hline
			\end{tabular}
			\label{m1}
		\end{center}
	\end{table}
	
	\section{Написать функцию longer\_then от двух списков-аргументов, которая возвращает Т, если первый аргумент имеет большую длину.}
	\textit{(defun longer\_than (a b) (cond ((> (length a) (length b)) T)))}
	
	Пример:
	\begin{itemize}
		\item (longer\_than '(a b c d) '(a b c)) = T
		\item (longer\_than '(f ss) '(r w q)) = Nil
		\item (longer\_than '(1 (2 (3 (4)))) '(1 (2 (3)))) = Nil
	\end{itemize}
	
	\section{Каковы результаты вычисления следующих выражений?}
	
	\begin{table} [h!]
		\begin{center}
			\begin{tabular}{|l|l|}
				\hline
				{\bf  Выражение} &    {\bf Результат} \\
				\hline
				{(cons 3 (list 5 6))} & (3 5 6)\\
				\hline
				{(list 3 'from 9 'lives (- 9 3))} & (3 FROM 9 LIVES 6)\\
				\hline
				{((+ (length for 2 too)) (car '(21 22 23)))} &error: (+ (LENGTH FOR 2 TOO)) is not a function name\\
				\hline
				{(cdr ' (cons is short for ans))} & (IS SHORT FOR ANS)\\
				\hline
				{(car (list one two))} & error: variable ONE has no value\\
				\hline
				{(cons 3 '(list 5 6))} & (3 LIST 5 6)\\
				\hline
				{(car (list 'one 'two))} & ONE\\
				\hline
			\end{tabular}  
			\label{m2}
		\end{center}
	\end{table}
	
	
	\section*{Ответы на вопросы}
	
	\begin{enumerate}
		\item Базис - это минимально необходимый набор конструкций, с помощью которого можно реализовать задачу.
		\item Варианты классификаций функции LISP
		\begin{itemize}
			\item чистые(математические): принимают строго определенное число аргументов и возвращают одно значение
			\item формы: могут принимать разное число аргументов, в зависимости от чего по-разному себя ведет
			\item функционалы: принимают функциональные описания.
		\end{itemize}
		\item Классификация базовых функций Lisp
		\begin{itemize}
			\item функции-селекторы (функции доступа к данным): car, cdr
			\item функции-конструкторы: cons, list
			\item функции-предикаты: atom, Null, lisp и тю дю
			\item функции-сравнения: eq, eql, -, equl, equalp
		\end{itemize}
		\item Как работают car, cdr
		Функции car и cdr являются базисными функциями-селекторами.
		car принимает в качестве аргумента точечную пару или список. Функция возвращает голову списка. В случае точечной пары или непустого списка функция вернет первый элемент, в случае пустого списка - Nil. cdr также принимает в качестве аргумента точечную пару или список. Функция возвращает хвост списка, т. е. список, состоящий из всех элементов кроме первого. Если в списке меньше двух элементов, то функция возвращает Nil.
		\item Отличия list и cons
		Cons - базис языка, на вход принимает ровно два аргумента и создает одну списковую ячейку (расставляет указатели).
		List - написана на базе cons, принимает любое количество аргументов и создает список.
		Пример:
		(cons 'a (cons 'b (cons 'c Nil))) = (a b c)
		(list 'a 'b 'c) = (a b c)
	\end{enumerate}

	
\end{document}