\documentclass[a4paper,12pt]{article}

\usepackage[T2A]{fontenc}			
\usepackage[utf8]{inputenc}			
\usepackage[english,russian]{babel}	

\usepackage[
bookmarks=true, colorlinks=true, unicode=true,
urlcolor=black,linkcolor=black, anchorcolor=black,
citecolor=black, menucolor=black, filecolor=black,
]{hyperref}

\usepackage{color}
\usepackage{caption}
\DeclareCaptionFont{white}{\color{black}}
\DeclareCaptionFormat{listing}{\colorbox{white}{\parbox{\textwidth}{#1#2#3}}}

\usepackage{amsmath,amsfonts,amssymb,amsthm,mathtools} 
\usepackage{wasysym}

\usepackage[cache=false]{minted}

\usepackage{graphicx}
\usepackage{cmap}
\usepackage{indentfirst}

\usepackage{listings} 
\usepackage{fancyvrb}

\usepackage{geometry}
\geometry{left=2cm}
\geometry{right=1.5cm}
\geometry{top=1cm}
\geometry{bottom=2cm}

\usepackage{longtable}

\setlength{\parindent}{5ex}
\setlength{\parskip}{0.5em}

\usepackage{pgfplots}
\usetikzlibrary{datavisualization}
\usetikzlibrary{datavisualization.formats.functions}

\begin{document}
	\lstset{ %
		language=C,                 % выбор языка для подсветки (здесь это С)
		basicstyle=\small\sffamily, % размер и начертание шрифта для подсветки кода
		numbers=left,               % где поставить нумерацию строк (слева\справа)
		numberstyle=\tiny,           % размер шрифта для номеров строк
		stepnumber=1,                   % размер шага между двумя номерами строк
		numbersep=5pt,                % как далеко отстоят номера строк от подсвечиваемого кода
		backgroundcolor=\color{white}, % цвет фона подсветки - используем \usepackage{color}
		showspaces=false,            % показывать или нет пробелы специальными отступами
		showstringspaces=false,      % показывать или нет пробелы в строках
		showtabs=false,             % показывать или нет табуляцию в строках
		frame=single,              % рисовать рамку вокруг кода
		tabsize=2,                 % размер табуляции по умолчанию равен 2 пробелам
		captionpos=t,              % позиция заголовка вверху [t] или внизу [b] 
		breaklines=true,           % автоматически переносить строки (да\нет)
		breakatwhitespace=false, % переносить строки только если есть пробел
		escapeinside={\%*}{*)}   % если нужно добавить комментарии в коде
	}

	% Титульный лист
	\begin{figure}[h!]
		\begin{center}
			{\includegraphics[scale = 0.4]{titul.jpg}}
			\label{titul}
		\end{center}
	\end{figure}
	
	\vspace*{15mm} 
	
	\huge
	\begin{center}
		Дисциплина: <<Функциональное и логическое программирование>>
	\end{center}
	\vspace*{15mm} 	
	
	\begin{center}
		Исправление ошибок 17, 18, 19, 20 лабораторных работ
	\end{center}
	
	\vspace*{15mm} 	
	
	\large
	\begin{flushright}
		Студент: Левушкин И. К. \\
		Группа: ИУ7-62Б \\
		Преподаватели: Толпинская Н. Б., \\ Строганов Ю. В. \\
	\end{flushright}
	
	\vspace*{30mm}
	\begin{center}
		Москва, 2020 г.  
	\end{center}
	\thispagestyle{empty}
	
	
	\newpage

	\section*{Исправление ошибок 17-ой лабораторной работы}
	
	\subsection*{Замечание}
	
	\textit{А нельзя результат сразу определить в заголовке, а потом убедиться в его правильности! Исправьте тексты программ и одну табл.}
	
	\subsubsection*{Исправленные тексты программ}
	
	\textit{Реализация программы без использования отсечения}
	
	\begin{minted}{prolog}
	
	domains
	num1, num2, num3, result = integer
	predicates
	max_two(num1, num2, result).
	max_three(num1, num2, num3, result).
	clauses
	max_two(Num1, Num2, Num1) :- Num2 <= Num1.
	
	max_two(Num1, Num2, Num2) :- Num2 > Num1.
	
	max_three(Num1, Num2, Num3, Num1) :- Num1 >= Num2, Num1 >= Num3.
	
	max_three(Num1, Num2, Num3, Num2) :- Num2 > Num1, Num2 > Num3.
	
	max_three(Num1, Num2, Num3, Num3) :- Num3 >= Num2, Num3 > Num1.
	\end{minted}
	
	\textit{Реализация программы с использованием отсечения}
	
	\begin{minted}{prolog}
	
	domains
	num1, num2, num3, result = integer
	predicates
	max_two(num1, num2, result).
	max_three(num1, num2, num3, result).
	clauses
	max_two(Num1, Num2, Num1) :-
	Num1 > Num2, !.
	
	max_two(_, Num2, Num2).
	
	
	max_three(Num1, Num2, Num3, Num1) :-
	Num1 > Num2, Num1 > Num3, !.
	
	max_three(_, Num2, Num3, Num2) :-
	Num2 > Num3, !.
	
	max_three(_, _, Num3, Num3).
	\end{minted}
	
	\subsubsection*{Исправленная таблица для программы с использованием отсечения}
	
	\textit{Порядок работы системы для 4 примера:}
	
	\begin{minted}{prolog}
goal
	
	max_three(2, 2, 2, Result).
%Вывод:
	
	Result=2
	1 Solution
	\end{minted}
	
	\begin{center}
		\begin{longtable}[h!]{|p{0.05\linewidth}|p{0.25\linewidth}|p{ 0.3\linewidth}|p{ 0.3\linewidth}|}
			
			\hline
			
			{№ шага} & {Состояние 
				
				
				
				резольвенты, и 
				
				
				
				вывод: дальнейшие 
				
				
				
				действия (почему?)} & {Для каких термов 
				
				
				
				запускается алгоритм 
				
				
				
				унификации: Т1=Т2 и 
				
				
				
				каков {\bf результат} (и 
				
				
				
				подстановка)} & {Дальнейшие действия: 
				
				
				
				прямой ход или откат 
				
				
				
				(почему и к чему 
				
				
				
				приводит?)}\\
			
			\hline
			
			{1} & {max\_three(2, 2, 2, Result).} & {T1 = max\_three(2, 2, 2, Result);
				
				
				
				T2 = max\_two(Num1, Num2, Num1). 
				
				
				
				Неудача (функторы max\_three и max\_two не равны).} & {Прямой ход к следующему предложению.}\\
			
			\hline
			
			{2} & {max\_three(2, 2, 2, Result).} & {T1 = max\_three(2, 2, 2, Result);
				
				
				
				T2 = max\_two(\_, Num2, Num2). Неудача (функторы max\_three и max\_two не равны).} & {Прямой ход к следующему преложению.}\\
			
			\hline
			
			{3} & {max\_three(2, 2, 2, Result).} & {T1 = max\_three(2, 2, 2, Result);
				
				
				
				T2 = max\_three(Num1, Num2, Num3, Num1).
				
				
				
				Успех. Подстановка 2 = Num1, 2 = Num2, 2 = Num3, Result = Num1.} & {Прямой ход к 2 > 2.}\\
			
			\hline
			
			{4} & {2 > 2,
				
				
				
				2 > 2,
				
				
				
				Result = 2,
				
				!.} & {T1 = 2 > 2.
				
				Знак > имеет смысл сравнения, поскольку с обеих сторон от знака находятся конкретные значения. Неудача (2 !> 2).} & {Откат к предыдущему состоянию резольвенты: 
				
				max\_three(2, 2, 2, Result), Реконкретизация 2, 2, 2, Result.}\\
			\hline
			{5} & {max\_three(2, 2, 2, Result).} & {T1 = max\_three(2, 2, 2, Result);
				
				T2 = max\_three(\_, Num2, Num3, Num2).
				
				Успех. Подстановка 2 = Num2, 2 = Num3, Result = Num2.} & {Прямой ход к 2 > 2.}\\
			\hline
			{6} & {2 > 2,
				
				Result = 2,
				
				!.} & {T1 = 2 > 2.
				
				Знак > имеет смысл сравнения, поскольку с обеих сторон от знака находятся конкретные значения. Неудача (2 !> 2).} & {Откат к предыдущему состоянию резольвенты: 
				
				max\_three(2, 2, 2, Result), Реконкретизация 2, 2, Result.}\\
			
			\hline
			
			{7} & {max\_three(2, 2, 2, Result).} & {T1 = max\_three(2, 2, 2, Result);
				
				
				
				T2 = max\_three(\_, \_, Num3, Num3).
				
				
				
				Успех. Подстановка 2 = Num3, Result = Num3.} & {Прямой ход к Result = 2.}\\
			
			\hline
			
			{8} & {Result = 2.} & {T1 = Result = 2.
				
				
				
				Знак = имеет смысл присвоения, поскольку с одной из сторон от знака стоит свободная переменная. Подстановка Result = 2.} & {Вывод:
				
				
				
				Result = 2. Откат к предыдущему состоянию резольвенты:
				
				max\_three(2, 2, 2, Result), Реконкретизация 2, 2, 2, Result.}\\
			
			\hline
			
			{9} & {max\_three(2, 2, 2, Result).
				
				
				
				конец clauses,
				
				
				
				опустошение резольвенты,
				
				завершение работы.} & {} & {}\\
			
			\hline
			\label{m1}
		\end{longtable}
	\end{center}
	
	\subsection*{Замечание}
	
	\textit{Каково назначение использования алгоритма унификации?}
		
	\textit{Назначение алгоритма унификации заключается в попарном сопоставлении НЕТ термов и попытке построить для них общий пример. ЗАЧЕМ???}
	
	\subsubsection*{Исправление}
	
	Назначение использования алгоритма унификации двух термов состоит в том, чтобы подобрать нужное в данный момент правило.
	
	\subsection*{Замечание}
	
	Каков результат работы алгоритма унификации? 
	
	Результатом использования алгоритма унификации может быть успех или тупиковая ситуация (неудача).
	
	\textit{Результатом – не подстановка??}
	
	\subsubsection*{Исправление}
	
	Результатом использования алгоритма унификации может быть успех согласования базы знаний и вопроса, в качестве побочного эффекта формируется подстановка; или может быть тупиковая ситуация (неудача).
	
	\section*{Исправление ошибок 18-ой лабораторной работы}
	
	\subsection*{Замечание}
	
	
	\textit{factorial\_help: А нельзя в  1-м правиле Rezult определить в заголовке?}
	
	\textit{Factorial  А Help не всегда =1??? Можно прямо в терме!}

	\subsubsection*{Исправленный текст программы n!}
	
	\begin{minted}{prolog}
	domains
	
	num, help, result = integer
	
	predicates
	
	factorial(num, result)
	factorial_help(num, help, result)
	
	clauses
	
	factorial_help(1, Help, Help) :- !.
	
	factorial_help(Num, Help, Result) :- 
	Help1 = Help * Num, Num1 = Num - 1, 
	factorial_help(Num1, Help1, Result).
	
	
	factorial(0, 1) :-!.
	
	factorial(Num, Result) :- factorial_help(Num, 1, Result).
	\end{minted}
	
	\subsection*{Замечание}
	
	\textit{Таблица: T1 = factorial(0, Result);}
	
	\textit{T2 = factorial(0, 1). Успех.}
	
	\textit{Подстановка{ 0 = 0, Result = 1} константы в подстановку не заносятся!}
	
	\textit{И что на этом работа системы заканчивается???  ИСПРАВИТЬ!!}
	
	\subsubsection*{Исправленная таблица для программы n!}
	
	\textit{Порядок работы системы для 3 примера:}
	
	
	\begin{minted}{prolog}
goal
	factorial(0, Result).
%Вывод:
	Result=1
	1 Solution
	\end{minted}
	
	\begin{center}
		\begin{longtable}[h!]{|p{0.05\linewidth}|p{0.25\linewidth}|p{ 0.3\linewidth}|p{ 0.3\linewidth}|}
			\hline
			{№ шага} & {Состояние 
				
				резольвенты, и 
				
				вывод: дальнейшие 
				
				действия (почему?)} & {Для каких термов 
				
				запускается алгоритм 
				
				унификации: Т1=Т2 и 
				
				каков {\bf результат} (и 
				
				подстановка)} & {Дальнейшие действия: 
				
				прямой ход или откат 
				
				(почему и к чему 
				
				приводит?)}\\
			\hline
			{1} & {factorial(0, Result).} & {T1 = factorial(0, Result);
				
				T2 = factorial\_help(1, Help, Result).
				
				Неудача (функторы factorial и factorial\_help не равны).} & {Прямой ход к следующему предложению.}\\
			\hline
			{2} & {factorial(0, Result).} & {T1 = factorial(0, Result);
				
				T2 = factorial\_help(Num, Help, Result).
				
				Неудача (функторы factorial и factorial\_help не равны).} & {Прямой ход к следующему предложению.}\\
			\hline
			{3} & {factorial(0, Result).} & {T1 = factorial(0, Result);
				
				T2 = factorial(0, 1).
				
				0 = 0, Result = 1 успех (подобрано знание) => Подстановка \{Result = 1\}.} & {Проверка тела правила factorial(0, 1).}\\
			\hline
			{4} & {!} & {Удаление из памяти альтернативных путей унификации цели
				
				Успех - ответ - <<Result = 1>>, метка на правиле factorial(0, 1).} & {Других альтернатив нет => система завершает работу с единственным результатом - <<Result = 1>>.}\\
			\hline
			\label{m2}
		\end{longtable}
	\end{center}

	\subsection*{Замечание}
	
	\textit{fib\_help(\_, Second, Help, Num, Result) :   проверки  Help = Num, Result = Second можно перенести в заголовок!}
	
	\subsubsection*{Исправленный текст программы n-е число Фибоначчи}
	
	\begin{minted}{prolog}
	domains
	
	elem, num, help, result = integer
	
	predicates
	
	fib(num, result)
	fib_help(elem First, elem Second, help, num, result)
	
	clauses
	
	fib_help(_, Second, Num, Num, Second) :- !.
	
	fib_help(First, Second, Help, Num, Result) :- 
	New_second = First + Second,
	New_help = Help + 1, 
	fib_help(Second, New_second, New_help, Num, Result).
	
	
	fib(0, 1) :-!.
	
	fib(Num, Result) :- 
	fib_help(1, 1, 1, Num, Result).
	\end{minted}
	
	\section*{Исправление ошибок 19-ой лабораторной работы}
	
	\subsection*{Замечание}
	
	\textit{Не поняла: Где CLAUSES?}
	
	\subsubsection*{Исправление}
	
	Прошу прощения, при конвертации из pdf в doc clauses съехали на предыдущую строку
	
	\textit{Исправленный текст программы <<найти длину списка (по верхнему уровню)>>}
	
	\begin{minted}{prolog}
domains
	
	lst = integer*
	count, sum = integer
	predicates
	
	len_list(lst, count).
	len_list_help(lst, count Help, count Result).
clauses
	
	len_list_help([], Help, Help).
	len_list_help([_|T], Help, Answer) :- 
	Next_help = Help + 1, len_list_help(T, Next_help, Answer).
	
	len_list(List, Answer) :- len_list_help(List, 0, Answer).
	\end{minted}
	
	\subsection*{Замечание}
	
	\textit{найти сумму элементов числового списка, стоящих на нечетных позициях}
	
	\textit{А нельзя в заголовке выделить сразу два элемента?}
	
	\subsubsection*{Исправленный текст программы <<найти сумму элементов числового списка, стоящих на нечетных позициях исходного списка (нумерация от 0)>>}
	
	\begin{minted}{prolog}
domains
	
	lst = integer*
	count, sum = integer
	predicates
	
	sum_odd_list(lst, sum).
	sum_odd_list_help(lst, sum Help, sum Answer).
clauses
	
	sum_odd_list_help([], Help, Help).
	
	sum_odd_list_help([_|[]], Help, Help) :-!.
	
	sum_odd_list_help([_, Next_H|T], Help, Answer) :-
	New_help = Help + Next_H, 
	sum_odd_list_help(T, New_help, Answer).
	
	sum_odd_list(List, Answer) :- sum_odd_list_help(List, 0, Answer).
	\end{minted}
	
	
	\section*{Исправление ошибок 20-ой лабораторной работы}
	
	\subsection*{Замечание}
	
	\textit{сформировать список из элементов числового списка, больших заданного значения}
	
	\textit{list\_create\_help([], \_, Help, Help). А результат не пустой список?}
	
	\textit{А незя формирование нового списка перенести в заголовок и Help не нужен?}
	
	\subsubsection*{Исправленный текст программы <<сформировать список из элементов числового списка, больших заданного значения>>.}
	
	\begin{minted}{prolog}
domains
	
	lstI = integer*
	number = integer
predicates
	
	list_create(lstI, number, lstI)
clauses
	list_create([], _, []).
	
	list_create([H|T], Number, [H|T2]) :-
	H > Number, list_create(T, Number, T2), !.
	
	list_create([_|T], Number, T2) :-
	list_create(T, Number, T2).
	\end{minted}
	
	\subsection*{Замечание}
	
	\textit{Объясните обязательность  ! во 2-м правиле}
	
	\subsubsection*{Объяснение ! во 2-ом правиле}
	
	Предикат отсечения нужен во 2-ом правиле, чтобы отсечь бесперспективный путь доказательства, в данном случае это 3-е правило list\_create([\_|T], Number, T2), поскольку если элемент списка больше заданного значения, необходимо добавить его в результирующий список, применив правило 2, а третье правило служит для тех случаев, когда элемент списка не больше заданного значения, соответственно, применять 3-е правило, если 2-е правило успешно, не нужно.
	
	В противном же случае, мы получим несколько решений, вместо одного необходимого.
	
	\subsection*{Замечание}
	
	\textit{список из элементов, стоящих на нечетных позициях: Выделяйте срвзу по два элемента!}
	
	\subsubsection*{Исправленный текст программы <<сформировать список из элементов, стоящих на нечетных позициях исходного списка (нумерация от 0)>>}
	
	\begin{minted}{prolog}
domains
	
	lstI = integer*
	number = integer
predicates
	
	odd_list_help(lstI, lstI, lstI)
	odd_list(lstI, lstI)
clauses
	
	odd_list_help([], Help, Help).
	
	odd_list_help([_|[]], Help, Help) :-!.
	
	odd_list_help([_, Next_H|T], Help, Answer) :-
	odd_list_help(T,  [Next_H|Help], Answer).
	
	odd_list(List, Answer) :- odd_list_help(List, [], Answer).
	\end{minted}
	
	\subsection*{Замечание}
	
	\textit{удалить заданный элемент Зачем reverse?}
	
	\subsubsection*{Объяснение}
	
	Поскольку элементы последовательно добавляются в голову результирующего списка, полученный результат будет перевернутый. Поэтому, в правиле, предназначенном для выхода из рекурсии вызывается правило reverse(Help, Answer), которое переворачивает список.
	
	Но можно обойтись и без reverse.
	
	\subsubsection*{Исправленный текст программы <<удалить заданный элемент из списка (один или все вхождения)>>}
	
	\begin{minted}{prolog}
domains
	lstI = integer*
	number = integer
predicates
	delete_elem_from_list(lstI, number, lstI)
clauses
	delete_elem_from_list([], _, []).
	
	delete_elem_from_list([H|T], Number, [H|T3]) :-
	H <> Number, delete_elem_from_list(T, Number, T3),!.
	
	delete_elem_from_list([_|T], Number, T3) :-
	delete_elem_from_list(T, Number, T3).
	\end{minted}
	

\end{document}