\documentclass[a4paper,12pt]{article}

\usepackage[T2A]{fontenc}			
\usepackage[utf8]{inputenc}			
\usepackage[english,russian]{babel}	

\usepackage[
bookmarks=true, colorlinks=true, unicode=true,
urlcolor=black,linkcolor=black, anchorcolor=black,
citecolor=black, menucolor=black, filecolor=black,
]{hyperref}

\usepackage{color}
\usepackage{caption}
\DeclareCaptionFont{white}{\color{black}}
\DeclareCaptionFormat{listing}{\colorbox{white}{\parbox{\textwidth}{#1#2#3}}}
\captionsetup[lstlisting]{format=listing,labelfont=white,textfont=white}

\usepackage{amsmath,amsfonts,amssymb,amsthm,mathtools} 
\usepackage{wasysym}

\usepackage{graphicx}
%\usepackage[cache=false]{minted}
\usepackage{cmap}
\usepackage{indentfirst}

\usepackage{listings} 
\usepackage{fancyvrb}

\usepackage{geometry}
\geometry{left=2cm}
\geometry{right=1.5cm}
\geometry{top=1cm}
\geometry{bottom=2cm}

\setlength{\parindent}{5ex}
\setlength{\parskip}{0.5em}

\usepackage{pgfplots}
\usetikzlibrary{datavisualization}
\usetikzlibrary{datavisualization.formats.functions}

\begin{document}
	\lstset{ %
		language=C,                 % выбор языка для подсветки (здесь это С)
		basicstyle=\small\sffamily, % размер и начертание шрифта для подсветки кода
		numbers=left,               % где поставить нумерацию строк (слева\справа)
		numberstyle=\tiny,           % размер шрифта для номеров строк
		stepnumber=1,                   % размер шага между двумя номерами строк
		numbersep=5pt,                % как далеко отстоят номера строк от подсвечиваемого кода
		backgroundcolor=\color{white}, % цвет фона подсветки - используем \usepackage{color}
		showspaces=false,            % показывать или нет пробелы специальными отступами
		showstringspaces=false,      % показывать или нет пробелы в строках
		showtabs=false,             % показывать или нет табуляцию в строках
		frame=single,              % рисовать рамку вокруг кода
		tabsize=2,                 % размер табуляции по умолчанию равен 2 пробелам
		captionpos=t,              % позиция заголовка вверху [t] или внизу [b] 
		breaklines=true,           % автоматически переносить строки (да\нет)
		breakatwhitespace=false, % переносить строки только если есть пробел
		escapeinside={\%*}{*)}   % если нужно добавить комментарии в коде
	}
	
	% Титульный лист
	\begin{figure}[h!]
		\begin{center}
			{\includegraphics[scale = 0.4]{titul.jpg}}
			\label{titul}
		\end{center}
	\end{figure}
	
	\vspace*{15mm} 
	
	\huge
	\begin{center}
		Дисциплина: <<Функциональное и логическое программирование>>
	\end{center}
	\vspace*{15mm} 	
	
	\begin{center}
		Лабораторная работа №5
	\end{center}
	
	\vspace*{15mm} 	
	
	\large
	\begin{flushright}
		Студент: Левушкин И. К. \\
		Группа: ИУ7-62Б \\
		Преподаватели: Толпинская Н. Б., \\ Строганов Ю. В. \\
	\end{flushright}
	
	\vspace*{30mm}
	\begin{center}
		Москва, 2020 г.  
	\end{center}
	\thispagestyle{empty}
	
	
	\newpage
	
	\section{Написать функцию, которая принимает целое число и возвращает первое
четное число, не меньшее аргумента.
	}

	Ответ: \textit{(defun near\_even (ar) (
		cond ((oddp ar) (+ ar 1))
		(T ar)
		))}

	\section{Написать функципо, которая принимает число и возвращает число
того же знака, но с модулем на 1 больше модуля аргумента.
	}

	Ответ: \textit{(defun abs\_more (ar) (
		cond ((< ar 0) (- ar 1))
		(T (+ ar 1))
		))}

	\section{Написать функцию, которая принимает два числа и возвращает
список из этих чисел, расположенный по возрастанию.
	}
	
	Ответ: \textit{(defun list\_increase (ar1 ar2) (
		cond ((< ar1 ar2) (list ar1 ar2))
		(T (list ar2 ar1))
		))}

	\section{Написать функцию, которая принимает три числа и возвращает Т только
тогда, когда первое число расположенно между вторым и третьим.
	}
	
	Ответ: \textit{(defun middle (ar1 ar2 ar3) (
		cond ((and (< ar1 ar3) (> ar1 ar2)) T)
		(T Nil)
		))}

	\section{Каков результат вычисления следующих выражений?}
	
	\begin{table} [h!]
		\begin{center}
			\begin{tabular}{|l|l|}
				\hline
				{\bf  Выражение} &    {\bf Результат} \\
				\hline
				{(and 'fee 'fie 'foe)} & FOE\\
				\hline
				{(or nil 'fie 'foe)} & FIE\\
				\hline
				{(and (equal 'abc 'abc) 'yes)} & YES\\
				\hline
				{(or 'fee 'fie 'foe)} & FEE\\
				\hline
				{(and nil 'fie 'foe)} & NIL\\
				\hline
				{(or (equal 'abc 'abc) 'yes)} & T\\
				\hline
			\end{tabular}  
			\label{m1}
		\end{center}
	\end{table}
	
	\section{Написать предикат, который принимает два числа-аргумента и возвращает
Т, если первое число не меньше второго.
	}

	Ответ: \textit{(defun not\_less (ar1 ar2) (
		cond ((< ar1 ar2) Nil)
		(T T)
		))}

	\section{Какой из следующих двух вариантов предиката ошибочен и почему?}
	
	\begin{enumerate}
		\item (defun pred1 (x) 	(and (numberp x) (plusp x)))
		\item (defun pred2 (x) (and (plusp x)(numberp x)))
	\end{enumerate}

	Второй предикат неверн, так как проверка на то, что аргумент является числом происходит после проверки на положительность, что вызовет ошибку при подаче не числового аргумента.
	
	
	\section{Решить задачу 4, используя для ее решения конструкции IF, COND, AND/OR.}
	
	\textit{COND:}
	(defun middle\_cond (ar1 ar2 ar3) (
	cond ( 
	(< ar1 ar3) (cond (
	(> ar1 ar2) T
	) 
	(T Nil) 
	)
	)
	(T Nil)
	))
	\vspace*{15mm} 
	
	\textit{IF:}
	(defun middle\_if (ar1 ar2 ar3) (
	if (< ar1 ar3) (if (> ar1 ar2) T Nil) Nil
	))
	
	\vspace*{15mm} 
	
	\textit{AND:}
	(defun middle\_and (ar1 ar2 ar3) (
	and (< ar1 ar3) (> ar1 ar2)
	))
	
	\vspace*{15mm} 


\end{document}