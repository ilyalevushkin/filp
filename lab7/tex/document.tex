\documentclass[a4paper,12pt]{article}

\usepackage[T2A]{fontenc}			
\usepackage[utf8]{inputenc}			
\usepackage[english,russian]{babel}	

\usepackage[
bookmarks=true, colorlinks=true, unicode=true,
urlcolor=black,linkcolor=black, anchorcolor=black,
citecolor=black, menucolor=black, filecolor=black,
]{hyperref}

\usepackage{color}
\usepackage{caption}
\DeclareCaptionFont{white}{\color{black}}
\DeclareCaptionFormat{listing}{\colorbox{white}{\parbox{\textwidth}{#1#2#3}}}
\captionsetup[lstlisting]{format=listing,labelfont=white,textfont=white}

\usepackage{amsmath,amsfonts,amssymb,amsthm,mathtools} 
\usepackage{wasysym}

\usepackage{graphicx}
%\usepackage[cache=false]{minted}
\usepackage{cmap}
\usepackage{indentfirst}

\usepackage{listings} 
\usepackage{fancyvrb}

\usepackage{geometry}
\geometry{left=2cm}
\geometry{right=1.5cm}
\geometry{top=1cm}
\geometry{bottom=2cm}

\setlength{\parindent}{5ex}
\setlength{\parskip}{0.5em}

\usepackage{pgfplots}
\usetikzlibrary{datavisualization}
\usetikzlibrary{datavisualization.formats.functions}

\begin{document}
	\lstset{ %
		language=C,                 % выбор языка для подсветки (здесь это С)
		basicstyle=\small\sffamily, % размер и начертание шрифта для подсветки кода
		numbers=left,               % где поставить нумерацию строк (слева\справа)
		numberstyle=\tiny,           % размер шрифта для номеров строк
		stepnumber=1,                   % размер шага между двумя номерами строк
		numbersep=5pt,                % как далеко отстоят номера строк от подсвечиваемого кода
		backgroundcolor=\color{white}, % цвет фона подсветки - используем \usepackage{color}
		showspaces=false,            % показывать или нет пробелы специальными отступами
		showstringspaces=false,      % показывать или нет пробелы в строках
		showtabs=false,             % показывать или нет табуляцию в строках
		frame=single,              % рисовать рамку вокруг кода
		tabsize=2,                 % размер табуляции по умолчанию равен 2 пробелам
		captionpos=t,              % позиция заголовка вверху [t] или внизу [b] 
		breaklines=true,           % автоматически переносить строки (да\нет)
		breakatwhitespace=false, % переносить строки только если есть пробел
		escapeinside={\%*}{*)}   % если нужно добавить комментарии в коде
	}
	
	% Титульный лист
	\begin{figure}[h!]
		\begin{center}
			{\includegraphics[scale = 0.4]{titul.jpg}}
			\label{titul}
		\end{center}
	\end{figure}
	
	\vspace*{15mm} 
	
	\huge
	\begin{center}
		Дисциплина: <<Функциональное и логическое программирование>>
	\end{center}
	\vspace*{15mm} 	
	
	\begin{center}
		Лабораторная работа №7
	\end{center}
	
	\vspace*{15mm} 	
	
	\large
	\begin{flushright}
		Студент: Левушкин И. К. \\
		Группа: ИУ7-62Б \\
		Преподаватели: Толпинская Н. Б., \\ Строганов Ю. В. \\
	\end{flushright}
	
	\vspace*{30mm}
	\begin{center}
		Москва, 2020 г.  
	\end{center}
	\thispagestyle{empty}
	
	
	\newpage
	
	\section{Чем принципиально отличаются функции cons, list, append?}
	
	\begin{itemize}
		\item cons - создает списковую ячейку
		\item list - создает список из своих аргументов
		\item append - создает список добавлением к 1ому аргументу-списку всех последующих
	\end{itemize}

	\subsection{Каковы результаты вычисления следующих выражений?}
	
	Пусть \textit{(setq lst1 '( a b)) (setq lst2 '( c d))}
	
	\begin{table} [h!]
		\begin{center}
			\begin{tabular}{|l|l|}
				\hline
				{\bf  Выражение} &    {\bf Результат} \\
				\hline
				{(cons Ist1 Ist2)} & ((A B) C D)\\
				\hline
				{(list lst1 Ist2)} & ((A B) (C D))\\
				\hline
				{(append Ist1 Ist2)} & (A B C D)\\
				\hline
			\end{tabular}  
			\label{m1}
		\end{center}
	\end{table}
	
	\section{Каковы результаты вычисления следующих выражений?}
	
	\begin{table} [h!]
		\begin{center}
			\begin{tabular}{|l|l|}
				\hline
				{\bf  Выражение} &    {\bf Результат} \\
				\hline
				{(reverse Q)} & error:variable Q has no value\\
				\hline
				{(last ())} & NIL\\
				\hline
				{(reverse '(a))} & (A)\\
				\hline
				{(last '(a))} & (A)\\
				\hline
				{(reverse '((a b c)))} & ((A B C))\\
				\hline
				{(last '((a b c)))} & ((A B C))\\
				\hline
			\end{tabular}  
			\label{m2}
		\end{center}
	\end{table}
	
	\section{Написать, по крайней мере, два варианта функции, которая возвращает последний элемент своего списка-аргумента.}
	
	\section{Написать, по крайней мере, два варианта функции, которая возвращает свой список-аргумент без последнего элемента.}
	
	\section{Написать простой вариант игры в кости, в котором бросаются две правильные кости. }
	
	\subsection{Условие}
	
	Если сумма выпавших очков равна 7 или 11 -- выигрыш, если выпало (1,1) или (6,6) --- игрок право снова бросить кости, во всех остальных случаях ход переходит ко второму игроку, но запоминается сумма выпавших очков. Если второй игрок не выигрывает абсолютно, то выигрывает тот игрок, у которого больше очков. Результат игры и значения выпавших костей выводить на экран с помощью функции print.
	
	\subsection{Решение}
	
	
	

\end{document}