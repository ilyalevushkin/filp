\documentclass[a4paper,12pt]{article}

\usepackage[T2A]{fontenc}			
\usepackage[utf8]{inputenc}			
\usepackage[english,russian]{babel}	

\usepackage[
bookmarks=true, colorlinks=true, unicode=true,
urlcolor=black,linkcolor=black, anchorcolor=black,
citecolor=black, menucolor=black, filecolor=black,
]{hyperref}

\usepackage{color}
\usepackage{caption}
\DeclareCaptionFont{white}{\color{black}}
\DeclareCaptionFormat{listing}{\colorbox{white}{\parbox{\textwidth}{#1#2#3}}}
\captionsetup[lstlisting]{format=listing,labelfont=white,textfont=white}

\usepackage{amsmath,amsfonts,amssymb,amsthm,mathtools} 
\usepackage{wasysym}

\usepackage[cache=false]{minted}

\usepackage{graphicx}
%\usepackage[cache=false]{minted}
\usepackage{cmap}
\usepackage{indentfirst}

\usepackage{listings} 
\usepackage{fancyvrb}

\usepackage{geometry}
\geometry{left=2cm}
\geometry{right=1.5cm}
\geometry{top=1cm}
\geometry{bottom=2cm}

\usepackage{longtable}

\setlength{\parindent}{5ex}
\setlength{\parskip}{0.5em}

\usepackage{pgfplots}
\usetikzlibrary{datavisualization}
\usetikzlibrary{datavisualization.formats.functions}

\begin{document}
	\lstset{ %
		language=C,                 % выбор языка для подсветки (здесь это С)
		basicstyle=\small\sffamily, % размер и начертание шрифта для подсветки кода
		numbers=left,               % где поставить нумерацию строк (слева\справа)
		numberstyle=\tiny,           % размер шрифта для номеров строк
		stepnumber=1,                   % размер шага между двумя номерами строк
		numbersep=5pt,                % как далеко отстоят номера строк от подсвечиваемого кода
		backgroundcolor=\color{white}, % цвет фона подсветки - используем \usepackage{color}
		showspaces=false,            % показывать или нет пробелы специальными отступами
		showstringspaces=false,      % показывать или нет пробелы в строках
		showtabs=false,             % показывать или нет табуляцию в строках
		frame=single,              % рисовать рамку вокруг кода
		tabsize=2,                 % размер табуляции по умолчанию равен 2 пробелам
		captionpos=t,              % позиция заголовка вверху [t] или внизу [b] 
		breaklines=true,           % автоматически переносить строки (да\нет)
		breakatwhitespace=false, % переносить строки только если есть пробел
		escapeinside={\%*}{*)}   % если нужно добавить комментарии в коде
	}
	
	% Титульный лист
	\begin{figure}[h!]
		\begin{center}
			{\includegraphics[scale = 0.4]{titul.jpg}}
			\label{titul}
		\end{center}
	\end{figure}
	
	\vspace*{15mm} 
	
	\huge
	\begin{center}
		Дисциплина: <<Функциональное и логическое программирование>>
	\end{center}
	\vspace*{15mm} 	
	
	\begin{center}
		Лабораторная работа №14
	\end{center}
	
	\vspace*{15mm} 	
	
	\large
	\begin{flushright}
		Студент: Левушкин И. К. \\
		Группа: ИУ7-62Б \\
		Преподаватели: Толпинская Н. Б., \\ Строганов Ю. В. \\
	\end{flushright}
	
	\vspace*{30mm}
	\begin{center}
		Москва, 2020 г.  
	\end{center}
	\thispagestyle{empty}
	
	
	\newpage
	
	\section*{Цель работы}
	
	 Получить навыки построения модели предметной области, разработки и оформления программы на Prolog, изучить принципы, логику формирования программы и отдельные шаги выполнения программы на Prolog.
	
	\section*{Задачи работы}
	
	Приобрести навыки декларативного описания предметной области с использованием фактов и правил.
	Изучить способы использования термов, переменных, фактов и правил в программе на Prolog, принципы  и правила сопоставления и отождествления, порядок унификации.
	
	
	\section*{Задание}
	
	Используя  базу знаний, хранящую знания (лаб. 13):
	\begin{itemize}
		\item <<{\bf Телефонный справочник}>>: Фамилия, №тел, Адрес – структура (Город, Улица, №дома, №кв),
		\item <<{\bf Автомобили}>>: Фамилия владельца, Марка, Цвет, Стоимость, и др.,
		\item <<{\bf Вкладчики банков}>>: Фамилия, Банк, счет, сумма, др.
	\end{itemize}

	Владелец может иметь несколько телефонов, автомобилей, вкладов (Факты). В разных городах есть однофамильцы, в одном городе – фамилия уникальна.
	
	Используя {\bf конъюнктивное правило и простой вопрос}, обеспечить возможность поиска:
	
	По Марке и Цвету автомобиля найти Фамилию, Город, Телефон и Банки, в которых владелец автомобиля имеет вклады. Лишней информации не находить и не передавать!!!
	
	Владельцев может быть {\bf несколько} (не более 3-х), {\bf один} и {\bf ни одного}.
	\begin{enumerate}
		\item Для каждого из трех вариантов {\bf словесно подробно} описать порядок формирования ответа (в виде таблицы). При этом, указать – отметить моменты очередного запуска алгоритма унификации и полный результат его работы. Обосновать следующий шаг работы системы. Выписать унификаторы – подстановки. Указать моменты, причины и результат отката, если он есть.
		\item Для случая нескольких владельцев (2-х): 
		приведите примеры (таблицы) работы системы {\bf при разных порядках} следования в БЗ  процедур, и знаний в них: ({\bf <<Телефонный справочник>>, <<Автомобили>>, <<Вкладчики банков>>}, или: {\bf <<Автомобили>>, <<Вкладчики банков>>, <<Телефонный справочник>>}). Сделайте вывод: Одинаковы ли: множество работ и объем работ в разных случаях?
		\item Оформите 2 таблицы, демонстрирующие {\bf порядок работы алгоритма унификации} вопроса и подходящего заголовка правила (для двух случаев из пункта 2) и укажите результаты его работы: ответ и побочный эффект.

	\end{enumerate}

	\newpage

	\section*{Реализация программы}
	
	\begin{minted}{prolog}
domains
address = adr(symbol, symbol, integer, integer).

predicates
phone_list(symbol Sername, string Phone, address Address).
auto(symbol Sername, symbol Model, symbol Color, integer Cost, 
integer Probeg).
bank_list(symbol Sername, symbol Bank, integer Account, integer Money).

get_info_by_model_color(symbol Model, symbol Color, 
symbol Sername, string Phone, symbol Bank).

clauses
phone_list(levushkin, "89859771492", adr(moscow, kantemirovskaya, 5, 1)).
phone_list(samkov, "89899999", adr(chelyabinsk, pushkinskaya, 4, 2)).
phone_list(ryazanova, "8911911911", adr(moscow, baumanskaya, 9, 9)).

auto(levushkin, shkoda, orange, 600000, 10000).
auto(levushkin, volvo, grey, 3000000, 1000).
auto(samkov, volkswagen, pink, 1000000, 99999).
auto(samkov, bugatti, gold, 999999999, 1).
auto(ryazanova, bugatti, gold, 999999999, 1).


bank_list(levushkin, sberbank, 1111, 900000).
bank_list(samkov, sberbank, 2222, 100).
bank_list(ryazanova, tinkoff, 3333, 99999999).
bank_list(ryazanova, raiffeisen, 4444, 888888888).

get_info_by_model_color(Model, Color, Sername, Phone, Bank) :-
auto(Sername, Model, Color, _, _), 
bank_list(Sername, Bank, _, _),
phone_list(Sername, Phone, _).
	\end{minted}
	
	\newpage
	
	\section*{Тесты}
	
	\subsection*{Фамилия, город, телефонный номер, название банка по марке и цвету авто}
	
	\subsubsection*{1. Несколько вариантов ответа.}
	
	\begin{minted}{prolog}
goal
	get_info_by_model_color(bugatti, gold, Sername, Phone, Bank).
%Вывод:
	Sername=samkov, Phone=89899999, Bank=sberbank
	Sername=ryazanova, Phone=8911911911, Bank=tinkoff
	Sername=ryazanova, Phone=8911911911, Bank=raiffeisen
	3 Solutions
	\end{minted}
	
	\textit{Порядок поиска ответа на вопрос:}
	
	Вопрос будет сопоставляться с каждым предложением сверху вниз, пока не
	найдется подходящий функтор. В таблице описаны ключевые шаги последующего поиска.
	
	\begin{center}
	\begin{longtable}[h!]{|p{0.05\linewidth}|p{0.5\linewidth}|p{ 0.4\linewidth}|}
				\hline
				{\bf  № шага} & {\bf Сравниваемые термы; результат; подстановка, если есть} & {\bf Дальнейшие действия: прямой ход или откат (к чему приводит?)} \\
				\hline
				{13} & {Сравнение get\_info\_by\_model\_color
					
					(bugatti, gold, Sername, Phone, Bank) с get\_info\_by\_model\_color(Model, Color, Sername, Phone, Bank). Успех. Подстановка Model=bugatti, Color=gold.} & {Прямой ход:
				
			auto(Sername, bugatti, gold, \_, \_). Сравнениваем сверху вниз -> первые 3 терма не подходят (разные названия функторов - phone\_list).}\\
				\hline
				{17} & {Сравнение auto(Sername, bugatti, gold, \_, \_) с auto(levushkin, shkoda, orange, \_, \_). Неудача (bugatti $\neq$ shkoda).} & {Откат к auto(Sername, bugatti, gold, \_, \_).}\\
				\hline
				{18} & {Сравнение auto(Sername, bugatti, gold, \_, \_) с auto(levushkin, volvo, grey, \_, \_). Неудача (bugatti $\neq$ volvo).} & {Откат к auto(Sername, bugatti, gold, \_, \_).}\\
				\hline
				{19} & {Сравнение auto(Sername, bugatti, gold, \_, \_) с auto(samkov, volkswagen, pink, \_, \_). Неудача (bugatti $\neq$ volkswagen).} & {Откат к auto(Sername, bugatti, gold, \_, \_).}\\
				\hline
				{20} & {Сравнение auto(Sername, bugatti, gold, \_, \_) с auto(samkov, bugatti, gold, \_, \_). Успех. Подстановка Sername=samkov.} & {Прямой ход:
					
					 bank\_list(samkov, Bank, \_, \_). Сравнение сверху вниз. Первые 8 термов не подходят (phone\_list, auto).}\\
				\hline
				{29} & {Сравнение bank\_list(samkov, Bank, \_, \_) с bank\_list(levushkin, sberbank, \_, \_). Неудача (samkov $\neq$ levushkin).} & {Откат к bank\_list(samkov, Bank, \_, \_).}\\
				\hline
				{30} & {Сравнение bank\_list(samkov, Bank, \_, \_) с bank\_list(samkov, sberbank, \_, \_). Успех. Подстановка Bank=sberbank.} & {Прямой ход:
				
					phone\_list(samkov, Phone, \_).}\\
				\hline
				{31} & {Сравнение phone\_list(samkov, Phone, \_) и phone\_list(levushkin, <<89859771492>>, \_). Неудача (samkov $\neq$ levushkin).} & {Откат к phone\_list(samkov, Phone, \_).}\\
				\hline
				{32} & {Сравнение phone\_list(samkov, Phone, \_) и phone\_list(samkov, <<89899999>>, \_). Успех. Подстановка Phone=<<89899999>>.} & {Вывод:
				
					Sername=samkov, Phone=<<89899999>>, Bank=sberbank.
				
			 Откат к phone\_list(samkov, Phone, \_).}\\
				\hline
				{33} & {Сравнение phone\_list(samkov, Phone, \_) и phone\_list(ryazanova, <<8911911911>>, \_). Неудача (samkov $\neq$ ryazanova).} & {Откат к phone\_list(samkov, Phone, \_). Следующие 10 термов не подходят (auto, bank\_list, get\_info\_by\_model\_color). Откат к bank\_list(samkov, Bank, \_, \_). Следующие 3 терма не подходят (bank\_list, get\_info\_by\_model\_color). Откат к auto(Sername, bugatti, gold, \_, \_).}\\
				\hline
				{47} & {Сравнение auto(Sername, bugatti, gold, \_, \_) с auto(ryazanova, bugatti, gold, \_, \_). Успех. Подстановка Sername=ryazanova.} & {Прямой ход:
				
					bank\_list(ryazanova, Bank, \_, \_). Следующие 8 термов не подходят (phone\_list, auto).}\\
				\hline
				{56} & {Сравнение bank\_list(ryazanova, Bank, \_, \_) с bank\_list(levushkin, sberbank, \_, \_). Неудача (ryazanova $\neq$ levushkin).} & {Откат к bank\_list(ryazanova, Bank, \_, \_).}\\
				\hline
				{57} & {Сравнение bank\_list(ryazanova, Bank, \_, \_) с bank\_list(samkov, sberbank, \_, \_). Неудача (ryazanova $\neq$ samkov)} & {Откат к bank\_list(ryazanova, Bank, \_, \_).}\\
				\hline
				{58} & {Сравнение bank\_list(ryazanova, Bank, \_, \_) с bank\_list(ryazanova, tinkoff, \_, \_). Успех. Подстановка Bank=tinkoff.} & {Прямой ход:
				
					phone\_list(ryazanova, Phone, \_).}\\
				\hline
				{59} & {Сравнение phone\_list(ryazanova, Phone, \_) с phone\_list(levushkin, <<89859771492>>, \_). Неудача (ryazanova $\neq$ levushkin).} & {Откат к phone\_list(ryazanova, Phone, \_).}\\
				\hline
				{60} & {Сравнение phone\_list(ryazanova, Phone, \_) с phone\_list(samkov, <<89899999>>, \_). Неудача (ryazanova $\neq$ samkov)} & {Откат к phone\_list(ryazanova, Phone, \_).}\\
				\hline
				{61} & {Сравнение phone\_list(ryazanova, Phone, \_) с phone\_list(ryazanova, <<8911911911>>, \_). Успех. Подстановка Phone=<<8911911911>>.} & {Вывод:
				
					Sername=ryazanova, Phone=89899999, Bank=tinkoff.
				
					Откат к phone\_list(ryazanova, Phone, \_). Следующие 10 термов не подходят (auto, bank\_list, get\_info\_by\_model\_color). Откат к bank\_list(ryazanova, Bank, \_, \_).}\\
				\hline
				{72} & {Сравнение bank\_list(ryazanova, Bank, \_, \_) c bank\_list(ryazanova, raiffeisen, \_, \_). Успех. Подстановка Bank=raiffeisen.} & {Прямой ход:
				
					phone\_list(ryazanova, Phone, \_).}\\
				\hline
				{73} & {Сравнение phone\_list(ryazanova, Phone, \_) с phone\_list(levushkin, <<89859771492>>, \_). Неудача (ryazanova $\neq$ levushkin).} & {Откат к phone\_list(ryazanova, Phone, \_).}\\
				\hline
				{74} & {Сравнение phone\_list(ryazanova, Phone, \_) с phone\_list(samkov, <<89899999>>, \_). Неудача (ryazanova $\neq$ samkov)} & {Откат к phone\_list(ryazanova, Phone, \_).}\\
				\hline
				{75} & {Сравнение phone\_list(ryazanova, Phone, \_) с phone\_list(ryazanova, <<8911911911>>, \_). Успех. Подстановка Phone=<<8911911911>>.} & {Вывод:
					
					Sername=ryazanova, Phone=89899999, Bank=raiffeisen.
					
					Откат к phone\_list(ryazanova, Phone, \_). Следующие 10 термов не подходят (auto, bank\_list, get\_info\_by\_model\_color). Откат к bank\_list(ryazanova, Bank, \_, \_). Следующий 1 терм не подходит (get\_info\_by\_model\_color). Откат к auto(Sername, bugatti, gold, \_, \_). Следующие 5 термов не подходят (bank\_list, get\_info\_by\_model\_color). Откат к get\_info\_by\_model\_color. Завершение работы (91 шаг)}\\
				\hline
			\label{m1}
	\end{longtable}
	\end{center}

	\newpage

	\subsubsection*{2. Один ответ.}
	
	\begin{minted}{prolog}
goal
	get_info_by_model_color(shkoda, orange, Sername, Phone, Bank).
%Вывод:
	Sername=levushkin, Phone=89859771492, Bank=sberbank
	1 Solution
	\end{minted}
	
	\textit{Порядок поиска ответа на вопрос:}
	
	Вопрос будет сопоставляться с каждым предложением сверху вниз, пока не
	найдется подходящий функтор. В таблице описаны ключевые шаги последующего поиска.
	
	\begin{center}
		\begin{longtable}[h!]{|p{0.05\linewidth}|p{0.5\linewidth}|p{ 0.4\linewidth}|}
				\hline
				{\bf  № шага} & {\bf Сравниваемые термы; результат; подстановка, если есть} & {\bf Дальнейшие действия: прямой ход или откат (к чему приводит?)} \\
				\hline
				{13} & {Сравнение get\_info\_by\_model\_color
				
					(shkoda, orange, Sername, Phone, Bank) с get\_info\_by\_model\_color(Model, Color, Sername, Phone, Bank). Успех. Подстановка Model=shkoda, Color=orange.} & {Прямой ход:
				
						auto(Sername, shkoda, orange, \_, \_). Сравниваем сверху вниз -> первые 3 терма не подходят (разные названия функторов - phone\_list).}\\
				\hline
				{17} & {Сравнение auto(Sername, shkoda, orange, \_, \_) с auto(levushkin, shkoda, orange, \_, \_). Успех. Подстановка Sername=levushkin.} & {Прямой ход:
				
					bank\_list(levushkin, Bank, \_, \_). Сравнение сверху вниз. Первые 8 термов не подходят (phone\_list, auto).}\\
				\hline
				{26} & {Сравнение bank\_list(levushkin, Bank, \_, \_) с bank\_list(levushkin, sberbank, \_, \_). Успех. Подстановка Bank=sberbank.} & {Прямой ход:
				
					phone\_list(levushkin, Phone, \_).}\\
				\hline
				{27} & {Сравнение phone\_list(levushkin, Phone, \_) с phone\_list(levushkin, <<89859771492>>, \_). Успех. Подстановка (Phone=<<89859771492>>).} & {Вывод:
				
					Sername=levushkin, Phone=<<89859771492>>, Bank=sberbank.
				
					Откат к phone\_list(levushkin, Phone, \_).}\\
				\hline
				{28} & {Сравнение phone\_list(levushkin, Phone, \_) с phone\_list(samkov, <<89899999>>, \_). Неудача (levushkin $\neq$ samkov).} & {Откат к phone\_list(levushkin, Phone, \_).}\\
				\hline
				{29} & {Сравнение phone\_list(levushkin, Phone, \_) с phone\_list(ryazanova, <<89899999>>, \_). Неудача (levushkin $\neq$ ryazanova).} & {Откат к phone\_list(levushkin, Phone, \_). Следующие 10 термов не подходят (auto, bank\_list, get\_info\_by\_model\_color). Откат к bank\_list(levushkin, Bank, \_, \_).}\\
				\hline
				{40} & {Сравнение bank\_list(levushkin, Bank, \_, \_) с bank\_list(samkov, sberbank, \_, \_). Неудача (levushkin $\neq$ samkov).} & {Откат к bank\_list(levushkin, Bank, \_, \_).}\\
				\hline
				{41} & {Сравнение bank\_list(levushkin, Bank, \_, \_) с bank\_list(ryazanova, tinkoff, \_, \_). Неудача (levushkin $\neq$ ryazanova).} & {Откат к bank\_list(levushkin, Bank, \_, \_).}\\
				\hline
				{42} & {Сравнение bank\_list(levushkin, Bank, \_, \_) с bank\_list(ryazanova, raiffeisen, \_, \_). Неудача (levushkin $\neq$ ryazanova).} & {Откат к bank\_list(levushkin, Bank, \_, \_). Следующий 1 терм не подходит (get\_info\_by\_model\_color). Откат к auto(Sername, shkoda, orange, \_, \_).}\\
				\hline
				{44} & {Сравнение auto(Sername, shkoda, orange, \_, \_) с auto(levushkin, volvo, grey, \_, \_). Неудача (shkoda $\neq$ volvo).} & {Откат к auto(Sername, shkoda, orange, \_, \_).}\\
				\hline
				{45} & {Сравнение auto(Sername, shkoda, orange, \_, \_) с auto(samkov, volkswagen, pink, \_, \_). Неудача (shkoda $\neq$ volkswagen).} & {Откат к auto(Sername, shkoda, orange, \_, \_). }\\
				\hline
				{46} & {Сравнение auto(Sername, shkoda, orange, \_, \_) с auto(samkov, bugatti, gold, \_, \_). Неудача (shkoda $\neq$ bugatti).} & {Откат к auto(Sername, shkoda, orange, \_, \_). }\\
				\hline
				{47} & {Сравнение auto(Sername, shkoda, orange, \_, \_) с auto(ryazanova, bugatti, gold, \_, \_). Неудача (shkoda $\neq$ bugatti).} & {Откат к auto(Sername, shkoda, orange, \_, \_). Следующие 5 термов не подходят (bank\_list, get\_info\_by\_model\_color). Откат к get\_info\_by\_model\_color. Завершение работы (52 шага).}\\
				\hline
			\label{m2}
	\end{longtable}
\end{center}

	\newpage

	\subsubsection*{3. Ни одного ответа.}
	
	\begin{minted}{prolog}
goal
	get_info_by_model_color(volvo, dark, Sername, Phone, Bank).
%Вывод:
	No Solution
	\end{minted}
	
	\textit{Порядок поиска ответа на вопрос:}
	
	Вопрос будет сопоставляться с каждым предложением сверху вниз, пока не
	найдется подходящий функтор. В таблице описаны ключевые шаги последующего поиска.
	
	\begin{center}
		\begin{longtable}[h!]{|p{0.05\linewidth}|p{0.5\linewidth}|p{ 0.4\linewidth}|}
				\hline
				{\bf  № шага} & {\bf Сравниваемые термы; результат; подстановка, если есть} & {\bf Дальнейшие действия: прямой ход или откат (к чему приводит?)} \\
				\hline
				{13} & {Сравнение get\_info\_by\_model\_color(volvo, dark, Sername, Phone, Bank) с get\_info\_by\_model\_color(Model, Color, Sername, Phone, Bank). Успех. Подстановка Model=volvo, Color=dark.} & {Прямой ход:
				
					auto(Sername, volvo, dark, \_, \_). Сравниваем сверху вниз -> первые 3 терма не подходят (разные названия функторов - phone\_list).}\\
				\hline
				{17} & {Сравнение auto(Sername, volvo, dark, \_, \_) с auto(levushkin, shkoda, orange, \_, \_). Неудача (volvo $\neq$ shkoda).} & {Откат к auto(Sername, volvo, dark, \_, \_).}\\
				\hline
				{18} & {Сравнение auto(Sername, volvo, dark, \_, \_) с auto(levushkin, volvo, grey, \_, \_). Неудача (dark $\neq$ grey).} & {Откат к auto(Sername, volvo, dark, \_, \_).}\\
				\hline
				{19} & {Сравнение auto(Sername, volvo, dark, \_, \_) с auto(samkov, volkswagen, pink, \_, \_). Неудача (volvo $\neq$ volkswagen).} & {Откат к auto(Sername, volvo, dark, \_, \_).}\\
				\hline
				{20} & {Сравнение auto(Sername, volvo, dark, \_, \_) с auto(samkov, bugatti, gold, \_, \_). Неудача (volvo $\neq$ bugatti).} & {Откат к auto(Sername, volvo, dark, \_, \_).}\\
				\hline
				{21} & {Сравнение auto(Sername, volvo, dark, \_, \_) с auto(ryazanova, bugatti, gold, \_, \_). Неудача (volvo $\neq$ ryazanova).} & {Откат к auto(Sername, volvo, dark, \_, \_). Следующие 5 термов не подходят (bank\_list, get\_info\_by\_model\_color). Откат к get\_info\_by\_model\_color. Завершение работы (26 шагов).}\\
				\hline
			\label{m3}
		\end{longtable}
	\end{center}
	
	\newpage
	
	\section*{Порядок работы алгоритма унификации при изменении порядка следования в Базе Знаний процедур и знаний в них.}
	
	Prolog обрабатывает правило в порядке следования предикатов в его теле, а не в базе знаний, следовательно, их порядок в БЗ не влияет ни на ход работы, ни, тем более, на результат. Таблицы, соответственно, полностью совпадут.
	
	\textit{Ниже приведена таблица, демонстрирующая порядок работы алгоритма унификации:}
	
	\begin{center}
		\begin{longtable}[h!]{|p{0.025\linewidth}|p{0.2\linewidth}|p{ 0.3\linewidth}|p{ 0.025\linewidth}|p{ 0.3\linewidth}|}
			\hline
			{\bf  № шага} & {\bf Результирую-
				
				щая ячейка} & {\bf Рабочее поле} & {\bf П. алг.} & {\bf Стек}\\
			\hline
			{0} & {} & {} & {1} & {get\_info\_by\_
				model\_color(bugatti, gold, Sername, Phone, Bank)=get\_info\_by\_
				mode\_color(Model, Color, Sername, Phone, Bank):- auto(Sername, Model, Color, \_, \_), bank\_list(Sername, Bank, \_, \_), phone\_list(Sername, Phone, \_).}\\
			\hline
			{1} & {} & {get\_info\_by\_
				model\_color(bugatti, gold, Sername, Phone, Bank)=get\_info\_by\_
				mode\_color(Model, Color, Sername, Phone, Bank):- auto(Sername, Model, Color, \_, \_), bank\_list(Sername, Bank, \_, \_), phone\_list(Sername, Phone, \_).} & {e} & {auto(Sername, bugatti, gold, \_, \_) = auto(samkov, bugatti, gold, \_, \_),
			
				auto(Sername, bugatti, gold, \_, \_) = auto(ryazanova, bugatti, gold, \_, \_),
			
				bank\_list(Sername, Bank, \_, \_),
			
				phone\_list(Sername, Phone, \_, \_).}\\
			\hline
			{2} & {Sername = samkov,
			
		Sername = ryazanova} & {auto(Sername, bugatti, gold, \_, \_) = auto(samkov, bugatti, gold, \_, \_),
			
				auto(Sername, bugatti, gold, \_, \_) = auto(ryazanova, bugatti, gold, \_, \_).} & {e} & {bank\_list(samkov, Bank, \_, \_) = bank\_list(samkov, sberbank, \_, \_),
			
				bank\_list(ryazanova, Bank, \_, \_) = bank\_list(ryazanova, tinkoff, \_, \_),
				
				bank\_list(ryazanova, Bank, \_, \_) = bank\_list(ryazanova, raiffeisen, \_, \_),
				
				phone\_list(samkov, Phone, \_, \_),
				
				phone\_list(ryazanova, Phone, \_, \_).}\\
			\hline
			{3} & {Sername = samkov,
				
				Bank = sberbank,
				
				Sername = ryazanova,
			
		Bank = tinkoff,
	
		Bank = raiffeisen.} & {bank\_list(samkov, Bank, \_, \_) = bank\_list(samkov, sberbank, \_, \_),
				
				bank\_list(ryazanova, Bank, \_, \_) = bank\_list(ryazanova, tinkoff, \_, \_),
			
		bank\_list(ryazanova, Bank, \_, \_) = bank\_list(ryazanova, raiffeisen, \_, \_).} & {e} & {phone\_list(samkov, Phone, \_, \_) = phone\_list(samkov, <<89899999>>, \_, \_),
		
		phone\_list(ryazanova, Phone, \_, \_) = phone\_list(ryazanova, <<8911911911>>, \_, \_).}\\
			\hline
			{4} & {Sername = samkov,
				
				Bank = sberbank,
				
				Sername = ryazanova,
				
				Bank = tinkoff,
				
				Bank = raiffeisen.} & {phone\_list(samkov, Phone, \_, \_) = phone\_list(samkov, <<89899999>>, \_, \_),
				
				phone\_list(ryazanova, Phone, \_, \_) = phone\_list(ryazanova, <<8911911911>>, \_, \_).} & {e} & {}\\
			\hline
			{Вы-
				
				вод:} & {\bf подстановка} & {Т. к. стек пуст - {\bf успех} и в рез. ячейнке подстановка} & {} & {}\\
			\hline
			\label{m4}
		\end{longtable}
	\end{center}
	
	\subsection*{Выводы}
	
	При разном следовании знаний и процедур количество операций не меняется. Поменяется лишь порядок сравнения. Это связано с тем, что (если не считать оптимизацию) алгоритм унификации реализует полный перебор.
	
	
	\newpage
	
	\section*{Ответы на вопросы}
	
	\subsection*{1.	В какой части правила сформулировано знание? Это знание о чем, с формальной точки зрения?}
	
	Заголовок содержит отдельное знание о предметной области (составной
	терм), а тело содержит условия истинности этого знания.
	Заголовок, как составной терм $f(t_1, t_2, ..., t_m)$, содержит знание о том, что
	между аргументами: $t_1, t_2, ..., t_m$ существует отношение (взаимосвязь, взаимозависимость). А имя этого отношения – это $f$.
	
	\subsection*{2.	Что такое процедура?}
	
	Процедура – совокупность правил, заголовки которых имеют одно и то же
	имя и одну и ту же арность, т.е. это совокупность правил, описывающих одно
	определенное отношение.
	
	\subsection*{3.	Сколько в БЗ текущего задания процедур?}
	
	В текущей БЗ 4 процедуры. Они описаны в predicates.
	
	\subsection*{4.	Что такое пример терма, это частный случай терма, пример? Как строится пример? }
	
	Пример терма – результат подстановки конкретных значений в предикат,
	частный случай предиката. Строится после того, как задан вопрос. Хранится
	до окончания работы программы.
	
	\subsection*{5.	Что такое наиболее общий пример?}
	
	Наиболее общий унификатор двух термов – унификатор, соответствующий
	наиболее общему их примеру.
	
	\subsection*{6.	Назначение и результат работы алгоритма унификации. Что значит двунаправленная передача параметров при работе алгоритма унификации, поясните на примере одного из случаев пункта  3.}
	
	Работа алгоритма унификации заключается в попарном сопоставлении термов и попытке построить для них общий пример. Алгоритм унификации производит двунаправленную передачу параметров процедурам. Двунаправленная
	передача параметров при работе алгоритма унификации – передача этих самых
	параметров извне в программу для дальнейшего использования или из программы во внешний мир (например, значение параметра, который нас интересует).
	
	\subsection*{7.	В каком случае запускается механизм отката?}
	
	Механизм отката запускается в случае, когда унификация завершается тупиковой ситуацией или неудачей. При этом происходит откат к предыдущему
	шагу.
	
	\subsection*{8.	Виды и назначение переменных в Prolog. Примеры из задания.  Почему использованы те или другие переменные (примеры из задания)?}
	
	\subsubsection*{Назначение переменных}
	
	Переменные предназначены для передачи значений в программе. Они являются частью процесса сопоставления и не являются <<хранилищем>> информации.
	
	Во время вычисления именованные переменные могут конкретизироваться
	(связываться с различными объектами), причем она может быть переконкретизирована, путем отката вычислительного процесса и отмены ранее проведенной конкретизации для нахождения новых решений.
	
	Анонимные переменные не могут быть связаны со значениями.
	
	\subsubsection*{Виды переменных}
	
	\begin{itemize}
		\item Именованные – обозначается комбинацией символов латинского алфавита,
		цифр и символа подчеркивания, начинающейся с прописной буквы или
		символа подчеркивания. Уникальны в рамках одного предложения.
		\item Анонимные – обозначаются символом подчеркивания. Любая анонимная
		переменная уникальна.
	\end{itemize}

	\subsubsection*{Пример из задания}
	
	\begin{minted}{prolog}
	get_info_by_model_color(Model, Color, Sername, Phone, Bank) :-
	auto(Sername, Model, Color, _, _), 
	bank_list(Sername, Bank, _, _),
	phone_list(Sername, Phone, _).
	\end{minted}
	
	Анонимные переменные используются, чтобы лишней информации не находить и не передавать. Именованные – нужны для достижения цели и конкретизации.
	
\end{document}